\documentclass{article}
\usepackage[utf8]{inputenc}
\usepackage[spanish]{babel}
\usepackage{hyperref}
\usepackage{graphicx}
\usepackage{listings}
\usepackage{xcolor}
\usepackage{amsmath}
\usepackage{textcomp}
\usepackage{longtable}
\usepackage{xltabular}
\usepackage{booktabs}
\usepackage{geometry}
\geometry{a4paper, margin=1in}

\lstset{
    basicstyle=\ttfamily\small,
    extendedchars=true,
    columns=fullflexible,
    keepspaces=true,
    literate={á}{{\'a}}1 {é}{{\'e}}1 {í}{{\'i}}1 {ó}{{\'o}}1 {ú}{{\'u}}1 {ñ}{{\~n}}1 {Á}{{\'A}}1 {É}{{\'E}}1 {Í}{{\'I}}1 {Ó}{{\'O}}1 {Ú}{{\'U}}1 {Ñ}{{\~N}}1 {¡}{{!`}}1 {¿}{{?`}}1 {°}{{$^\circ$}}1
}

\title{PLANT IDENTIFICATION SYSTEM}
\author{}
\date{}

\begin{document}

\maketitle
\section*{Plant Identification System}

\subsection*{Overview}
This project aims to create a system for identifying plants using image recognition technology. It utilizes an ESP32-CAM module to capture plant images and sends them to a Flask server for processing. The server then communicates with the PlantNet API to identify the plant species and returns the result to the user interface.

\subsection*{Components}
\begin{itemize}
    \item ESP32-CAM module: Captures plant images and sends them to the server.
    \item Flask server: Receives images from the ESP32-CAM, performs plant identification using the PlantNet API, and returns the result.
    \item PlantNet API: A plant identification API that takes an image as input and returns the identified plant species.
\end{itemize}

\subsection*{Requirements}
\begin{itemize}
    \item Hardware:
    \item ESP32-CAM module
    \item Software:
    \item Arduino IDE (for ESP32-CAM firmware)
    \item Python 3.x
    \item Flask
    \item Requests library
    \item PlantNet API key
\end{itemize}

\subsection*{Installation}
\begin{enumerate}
    \item \textbf{ESP32-CAM Firmware:}
\end{enumerate}
\begin{itemize}
    \item Upload the provided ESP32-CAM code to your ESP32-CAM module using the Arduino IDE. Make sure to update the Wi-Fi credentials (\texttt{ssid} and \texttt{password}) and the Flask server IP address (\texttt{http://192.168.xxx.xxx:33/identify}) in the code.
\end{itemize}

\begin{enumerate}
    \item \textbf{Flask Server:}
\end{enumerate}
\begin{itemize}
    \item Install Python 3.x and pip.
    \item Install Flask and Requests library using pip:
\end{itemize}
     \begin{lstlisting}
     pip install Flask requests
     
\end{lstlisting}

\begin{enumerate}
    \item \textbf{PlantNet API Key:}
\end{enumerate}
\begin{itemize}
    \item Sign up for an account on \href{https://plantnet.org/}{PlantNet} and obtain an API key.
\end{itemize}

\begin{enumerate}
    \item \textbf{Run the Server:}
\end{enumerate}
\begin{itemize}
    \item Replace the \texttt{API\_KEY} variable in \texttt{app.py} with your PlantNet API key.
    \item Start the Flask server:
\end{itemize}
     \begin{lstlisting}
     python app.py
     
\end{lstlisting}

\begin{enumerate}
    \item \textbf{Web Interface:}
\end{enumerate}
\begin{itemize}
    \item Access the web interface by navigating to \texttt{http://localhost:33} in your web browser.
\end{itemize}

\subsection*{Usage}
\begin{enumerate}
    \item Power on the ESP32-CAM module.
    \item Navigate to the web interface.
    \item Click on the \texttt{}Identify Plant'' button to capture an image.
    \item Wait for the identification process to complete.
    \item The identified plant species will be displayed on the web interface.
\end{enumerate}



\end{document}
